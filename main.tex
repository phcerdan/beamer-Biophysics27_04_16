\documentclass[9pt]{beamer}

\usetheme{metropolis}
\metroset{progressbar=frametitle}
\usepackage[absolute,overlay]{textpos}
\newcommand{\tikzmark}[1]{\tikz[remember picture] \node[coordinate] (#1) {#1};}
\usepackage{pgfpages}
% \setbeameroption{show notes on second screen}
\usepackage{multimedia}
\usepackage{booktabs}

\usepackage{subcaption}
% To fix in old caption: https://tex.stackexchange.com/questions/426088/texlive-pretest-2018-beamer-and-subfig-collide
\makeatletter
\let\@@magyar@captionfix\relax
\makeatother
\usepackage[square]{natbib} % Bibliography, citing: [\citet{storm_nonlinear_2005}]
\usepackage[scale=2]{ccicons}

\usepackage{pgfplots}
\usepgfplotslibrary{dateplot}
\graphicspath{{Figures/}}
\graphicspath{{Videos/}}
\title{Biopolymer Networks: Image Analysis, Reconstruction and Modeling}
\author{{\Large\textbf{Pablo Hernandez-Cerdan}}
  \emph{\newline \textit{Supervisor}:} Prof. M.A.K Williams%
  \emph{\newline \textit{Co-Supervisor}:} Prof. Geoff Jameson%
}
\date{Doctoral Oral Examination -- \today}
% \institute{PhD. Student \newline
%   Institute of Fundamental Sciences, Massey University\newline
%   MacDiarmid Institute for Advanced Materials and Nanotechnology\newline
%   Riddet Institute\newline
%   New Zealand
% }
% \titlegraphic{\hfill\includegraphics[height=1.5cm]{./logo.pdf}}
\begin{document}

\maketitle

% notes: with the table of contents displayes, take 1 min to explain the challenges you faced,
% talk about the path...
\begin{frame}
  \frametitle{Table of Contents}
  \setbeamertemplate{section in toc}[sections numbered]
  \tableofcontents[hideallsubsections]
\end{frame}

\section{Motivation}
\begin{frame}{Sources of Strain-Stiffening}
  \begin{itemize}
    \item \citep{storm_nonlinear_2005}: strain stiffening arises from non linearity of single chains.
      Under the assumption that the network is isotropic and homogeneous.
    \item \citep{onck_alternative_2005}, at the same time an alternative explanation of strain stiffening arises exclusively from
      network connectivity and relative orientation of the fibers.
  \end{itemize}
\end{frame}

\begin{frame}{Strain-Stiffening: Validity of Affine Deformation approximation in semi-flexible polymers}
  \note{ \citep{wilhelm_elasticity_2003} \citep{onck_alternative_2005}}
  \begin{columns}[onlytextwidth]
    \column{0.5\textwidth}
    \begin{exampleblock}{Affine transformations valid when any of this: }
      % \footnotesize{\textbf{In Homogeneous and Isotropic Network}}
      \begin{itemize}
        \item Isotropic and Homogeneous Network
        \item Stiff Chains
        \item Dense Networks
      \end{itemize}
      % \footnotesize{\citep{wilhelm_elasticity_2003}}
    \end{exampleblock}
    \begin{alertblock}{Non-Affine range when:}
      % \footnotesize{In Homogeneous and Isotropic Network}
      \begin{itemize}
        \item Anisotropic Network
        \item Compliant Chains
        \item Coarse Networks
      \end{itemize}
      \footnotesize{\citep{wilhelm_elasticity_2003}}
    \end{alertblock}
    \column{0.5\textwidth}
      \begin{figure}[htpb]
        \centering
        \includegraphics[width=0.99\textwidth]{./Figures/nonaffine.png}
        \caption*{Displacement of network points under affine and non-affine regimes. \footnotesize{\citep{basu_nonaffine_2011, wen_non-affine_2012}}}
      \end{figure}
      \note{ Difference between affine
        and non-affine deformation of a network under shear. Figure adopted from
        references}
  \end{columns}
\end{frame}
    % \column{0.5\textwidth}
    % \begin{figure}[htpb]
    %   \centering
    %   \includegraphics[width=0.9\textwidth]{./Figures/nonaffine.png}
    %   \caption*{\footnotesize{\citep{basu_nonaffine_2011, wen_non-affine_2012}}}
    % \end{figure}
    % \note{
    %   Difference between affine
    %   and non-affine deformation of a network under shear. Figure adopted from
    %   references
    % }
%   \end{columns}
%   \only<2->{
%   \begin{columns}[onlytextwidth]
%     \column{0.5\textwidth}
%     \begin{figure}[htpb]
%       \includegraphics[width=0.7\textwidth]{./Figures/nonaffine-head.png}
%       \caption*{\footnotesize{\textbf{L:} Molecular Weight, \textbf{c:} Concentration
%       \citep{head_mechanical_2005}}}
%     \end{figure}
%     \only<3->{
%       \column{0.5\textwidth}
%       \begin{exampleblock}{Pre-stress and architecture role: }
%         \footnotesize{
%         Affine approx. is valid at constrained networks}
%         \begin{itemize}
%           \item Initial pre-stress constrains networks.
%           \footnotesize{\citep{cioroianu_disorder_2015}}
%           \item<4> Role of \textbf{Anisotropy} or \textbf{Architecture}?
%         \end{itemize}
%       \end{exampleblock}
%     }
%   \note{
%     Molecular weight (L) and concentration (c). The solid line represents
%     the rigidity percolation transition, where rigidity first appears at
%     macroscopic level
%   }
% \end{columns}
% }
% \end{frame}
\begin{frame}
  \frametitle{Mechanical simulations use toy network architectures}
  \begin{figure}
    \begin{subfigure}[t]{0.33\textwidth}
      \centering
      \includegraphics[width=0.9\linewidth]{Figures/mikado_initial.png}
      \subcaption{Mikado model}
      \label{subfig:mikado}
    \end{subfigure}%
    \begin{subfigure}[t]{0.33\textwidth}
      \centering
      \includegraphics[width=0.9\linewidth]{Figures/fcc_lattice_wiki.png}
      \subcaption{Face-centered-cube (FCC) lattice}
      \label{subfig:fcc_lattice}
    \end{subfigure}%
    \begin{subfigure}[t]{0.33\textwidth}
      \centering
      \includegraphics[width=0.9\linewidth]{Figures/eightchain-simple.png}
      \subcaption{Eight-chain model}
      \label{subfig:eightchain}
    \end{subfigure}
    \caption{Different toy models used for mechanical simulations.}
    \label{fig:toy-models}
  \end{figure}
  \begin{center}
    \metroset{block=fill}
    \begin{exampleblock}{Toy models? We can use real architectures gathered from 3D microscopy data}
    \end{exampleblock}
  \end{center}
  % \metroset{block=fill}
  % \begin{exampleblock}{}
  % \begin{itemize}
  %   \item<1-> Going beyond the homogeneous and isotropic model of biopolymer networks: \textbf{Characterize the architecture} of connected filaments using images from different microscopy techniques depending on the size of the biopolymer.
  %   \item<2-> How different architectures do influent the \textbf{mechanical properties}? Are completely different biopolymers sharing similar geometries?
  %   \item<3-> Develop an open-source, free, user friendly, tested, and well documented software for others to use with microscopy images.
  % \end{itemize}
  % \end{exampleblock}
\end{frame}

\section{Goal of the thesis}
\begin{frame}
  \frametitle{Goals}
  % \metroset{block=fill}
  \begin{exampleblock}{}
  \begin{itemize}
    \item<1-> Extract the geometry of biopolymers networks from 3D images.
    \item<2-> Provide software as a tool-set, enabling others to extract and analyze networks.
  \end{itemize}
  \end{exampleblock}
\end{frame}
% \begin{frame}
%   \frametitle{Steps to characterize the network}
%   \begin{itemize}
%     \item Gathering the images. Microscopy.
%     \item Image analysis. Skeletonization.
%     \item Characterize the image with a graph representation.
%       \note
%       {
%         Grab the structure of a diluted enough network to be suceptible of the graph approach.
%         If it is too dense, probably the study of porosity would be more suitable
%       }
%   \end{itemize}
%       \begin{center}
%       \movie[height=0.4\textheight, width=0.65\textwidth, poster, autostart, loop]{}{./Videos/softwareShort.mp4}
%
%       \end{center}
% \end{frame}

\section{Challenges: Validating Microscopy Data}

\begin{frame}
  \frametitle{Scattering: SAXS and SANS}
  \begin{itemize}
    \item Study of peaks and fractality at different q values provides an \textbf{averaged} structural information (mesh size, width of single-chains)
    \item It does not require special sample preparation.
    \item Good statistics, fast, allow study of dynamics.
    \item \alert{But}, it does not provide explicit 3D structure of exact connectivity.
  \end{itemize}
    \begin{figure}[htpb]
      \includegraphics[width=0.25\textwidth]{./Figures/saxs_pectin.png}
      \caption*{\footnotesize{I vs q Pectin from SAXS}}
    \end{figure}
\end{frame}
\begin{frame}
  \frametitle{Microscopy: TEM and Confocal}
  To get detailed connectivity we need \textbf{3D microscopy}.
  \begin{columns}[T,onlytextwidth]
      \column{0.5\textwidth}
      \begin{exampleblock}{TEM tomography}
          \begin{itemize}
              \item Necessary to reach polysaccharides scale.
              \item Sample preparation is hard. Artifacts?
              \item In chapter 2 we concluded that network size features can be trusted.
          \end{itemize}
      \end{exampleblock}
      \column{0.5\textwidth}
      \begin{exampleblock}{Confocal}
          \begin{itemize}
              \item Suitable for most protein biopolymer networks.
              \item Sample preparation is more reliable.
          \end{itemize}
      \end{exampleblock}
  \end{columns}
  \begin{columns}[T,onlytextwidth]
  \column{0.5\textwidth}
      \centering\includegraphics[width=0.5\textwidth]{./Figures/image_tem.png}
  \column{0.5\textwidth}
      \centering\includegraphics[width=0.5\textwidth]{./Figures/image_confocal.png}
  \end{columns}
\end{frame}

\section{Challenges: Image Denoising and Skeletonization}
\begin{frame}[t]
  \frametitle{Image Denoising and Skeletonization}
    \vspace{0.8cm}
    \metroset{block=fill}
    \begin{exampleblock}{Wavelet denoising}
      Developed Isotropic Wavelets framework to remove noise, and aid with the segmentation in the extraction pipeline.
    \end{exampleblock}
    \vspace{0.8cm}
    \metroset{block=fill}
    \begin{exampleblock}{Skeletonization algorithms}
      \begin{enumerate}
        \item Developed state of art method from Digital Topology fields to get a thin skeleton representation of the biopolymer network.
      \end{enumerate}
    \end{exampleblock}
    \vspace{1cm}
    % \noindent\rule[0.1pt]{\linewidth}{0.01pt}
\end{frame}

\section{Extraction of Architecture and Graph Characterization}
\begin{frame}[t]
  \frametitle{From Images to Spatial Graphs}
    \begin{alertblock}{Goal:}
        Process raw image data to get a spatial graph representation of the network.
    \end{alertblock}
    \vspace{0.8cm}
    \metroset{block=fill}
    \begin{exampleblock}{Steps:}
      \begin{enumerate}
        \item \textbf{Pre-processing:} Enhance signal to noise ratio with pre-processing (denoise).
        \item \textbf{Segmentation:} Extract object of interest from image. Binary image.
        \item \textbf{Skeletonization:} Thin representation of the object.
      \end{enumerate}
    \end{exampleblock}
    \vspace{1cm}
    The skeleton will then be adapted into a \textbf{Graph} for characterization.
    % \noindent\rule[0.1pt]{\linewidth}{0.01pt}
\end{frame}



\begin{frame}
  \frametitle{Graphs: A Link to Complex Systems}
  A graph is a representation of connected components in terms of nodes and edges/links.

  \begin{columns}[onlytextwidth]
      \column{0.5\textwidth}
        \begin{exampleblock}{Spatial Network of biopolymers:}
          \begin{itemize}
            \item A node corresponds to a junction/crosslink of biopolymers.
            \item An edge represents the polymer-chain itself.
          \end{itemize}
        \end{exampleblock}
      \column{0.5\textwidth}
        \includegraphics[width=0.8\linewidth]{./Figures/spatial_graph.png}
  \end{columns}

  Graphs are a powerful tool connecting to the field of \textbf{Complex Systems}.

  We want to take advantage of all those existing algorithms for \textbf{characterization} of soft materials.

\end{frame}
\begin{frame}
  \frametitle{Characterization of graphs: Statistical Distributions}
  We are able to analyze the graph to gather statistical distributions of some properties:
  \vspace{0.3cm}
  \begin{columns}[onlytextwidth]
      \column{0.5\textwidth}
        \includegraphics[width=0.9\linewidth]{./Figures/distributions_maths.png}
      \column{0.5\textwidth}
        \includegraphics[width=0.9\linewidth]{./Figures/distributions_actin.png}
  \end{columns}
  \vspace{0.3cm}
  Protein and polysaccharides networks share the same \textbf{functional distributions} for
  degree, end-to-end distances, contour-lengths and angles.
\end{frame}
\begin{frame}
  \frametitle{Reconstruction in-silico}
  \begin{columns}[onlytextwidth]
      \column{0.41\textwidth}
      \metroset{block=fill}
      \begin{exampleblock}{Reconstruction algorithm}
         \begin{itemize}
           \item Also provided, as a exploratory tool for different networks architectures is the reconstruction algorithm.
           \item From the set of statistical distributions, it is able to generate the full geometry of a network which follows the targeted distributions.
         \end{itemize}
       \end{exampleblock}
      \column{0.59\textwidth}
        \includegraphics[width=0.99\textwidth]{./Figures/project_scheme.png}
  \end{columns}
\end{frame}

% \begin{frame}
%   \frametitle{Results: Proof of concept}
%   \footnotesize{
%   \uncover<1>{Three Networks analyzed:
%   \begin{columns}[onlytextwidth]
%       \column{0.5\textwidth}
%       \begin{itemize}
%       \item Polysaccharides using TEM: Pectin, Carrageenan
%       \end{itemize}
%       \column{0.5\textwidth}
%       \begin{itemize}
%           \item Proteins using Confocal: Actin (Collagen from literature \citep{lindstrom_finite-strain_2013} ).
%       \end{itemize}
%   \end{columns}
%   }
%   \uncover<2->{
%       Statistical distributions of local graph properties:
%       \alert{Showing same functional forms}
%
%       \vspace{1mm}
%   }
%   }
%   \vspace{0.5cm}
%   \centering\includegraphics[width=0.99\textwidth]{./Figures/results_poster.png}
% \end{frame}
\begin{frame}
  \frametitle{Conclusions}
  \begin{columns}[onlytextwidth]
      \column{0.60\textwidth}
      \metroset{block=fill}
      \begin{alertblock}{Exploring network geometries:}
          \begin{itemize}
              \item Provide software for reliably extract biopolymer networks from 3D images
              \item Link to complex network libraries for graph analysis. \textbf{Reconstruct in-silico networks from it}
              \item The tool-set developed provide soft-matter scientists to use these networks for simulations and characterization of materials.
          \end{itemize}
      \end{alertblock}
      \column{0.40\textwidth}
        \centering\includegraphics[width=\textwidth]{./Figures/network_model_fig.png}
  \end{columns}
\end{frame}
\plain{Thanks!}
\begin{frame}[allowframebreaks]

  \frametitle{References}

  \bibliography{bibliography}
  \bibliographystyle{plainnat}

\end{frame}
%%%%%%%%%%%%%%%%%%%%%%%%%%%%%%%%%%%%%%
% Others, left here just in case
%%%%%%%%%%%%%%%%%%%%%%%%%%%%%%%%%%%%%%

\begin{frame}
  \frametitle{The limit of the isotropic approximation}
  \begin{itemize}
    \item The power law behaviour has derived from universal slope of 1 \cite{mackintosh_bill} to \textbf{surprisingly(?)} a slope of  3/2. We think that this data require a further explanation beyond the assumption that collagen is somehow an exception in that universality claimed years ago.
    \item The shift to strain stiffening, or the beginning of a nematic phase of partially oriented fibers, might depend on architecture.
  \end{itemize}
\end{frame}

\begin{frame}{Acknowledgments}
  \vspace{0.3cm}
  \begin{columns}[onlytextwidth]
    \column{0.33\textwidth}
    \includegraphics[width=0.9\textwidth]{./Figures/people/macdiarmid.png}
    \column{0.33\textwidth}
    \includegraphics[width=0.9\textwidth]{./Figures/people/massey.png}
    \column{0.33\textwidth}
    \includegraphics[width=0.9\textwidth]{./Figures/people/riddet.png}
  \end{columns}
  \begin{columns}[onlytextwidth]
    \column{0.5\textwidth}
\begin{figure}[htpb]
  \centering
  \includegraphics[width=0.8\textwidth]{./Figures/people/bill.png}
  \caption*{Main Supervisor: \textbf{Prof. M.A.K Williams}}
\end{figure}
    \column{0.5\textwidth}
    \begin{figure}[htpb]
      \centering
      \includegraphics[width=0.3\textwidth]{./Figures/people/brad.png}
      \caption*{Dr. Brad Mansel, New Zealand}
    \end{figure}
  \end{columns}
  \begin{columns}[onlytextwidth]
    \column{0.45\textwidth}
    \begin{itemize}
      \item Andrew Leis. CSIRO Animal Health. Australia.
      \item Leif Lundin. SP Food and Bioscience. Sweden.
    \end{itemize}
    \column{0.45\textwidth}
    \textbf{Visit us in NZ or online at:}\newline
        \url{www.biophysics.ac.nz}
  \end{columns}
\end{frame}

\begin{frame}[t]
  \frametitle{Segmentation I}
    \alert{Goal:} Extract feature or region of interest from the raw image.

     Noise is the enemy.
     But also crowded environments from where only a specific feature is needed.

     \vspace{0.8cm}
     \metroset{block=fill}
     \begin{exampleblock}{Main Segmentation Methods:}
       \begin{itemize}
         \item \textbf{Threshold-Based:} Threshold at an intensity value. Entropy minimizing (Otsu) or other criteria.
         % \item \textbf{Region-Growing:} Local homogeneity of intensity criteria.
         \item \textbf{Enhancement Filter:} Local-phase studies or eigenvalue analysis of Hessian matrix.
         \item \textbf{Deformable-Model:} Curve evolving under vector field to fit structure. LevelSet (closed curve) or Snakes (open).
       \end{itemize}
     \end{exampleblock}
    % \noindent\rule[0.1pt]{\linewidth}{0.01pt}
\end{frame}

\begin{frame}[t]
  \frametitle{Segmentation in Confocal and TEM images of biopolymers}
     Need of preprocessing steps to reduce noise and enhance features of interest.
     \metroset{block=fill}
     \begin{exampleblock}{Pre-processing pipeline:}
     \begin{itemize}
       \item Multiscale characterization convolving image with gaussian operator of different sigmas/scales
       \item Denoise: Anisotropic edge-preserving filter.
       \item Enhancement Filter: Use a local-phase filter using a \textbf{monogenic-signal}, a N-D extension to the \textbf{analytic signal} used in modulation and demodulation of 1D communication signals.
     \end{itemize}
     \end{exampleblock}
     % \vspace{0.8cm}
    % \noindent\rule[0.1pt]{\linewidth}{0.01pt}
\end{frame}

\begin{frame}[t]
  \frametitle{Monogenic Signal: contribution to ITK}
    \begin{figure}[h]
      \centering
      \includegraphics[height=1cm]{./Figures/software_logos/itkLogo.png}
      \caption*{Insight Toolkit: Registration and Segmentation}
    \end{figure}
    I am implementing a monogenic signal filter to ITK, the insight toolkit for image segmentation based on \citep{chenouard_3d_2011}.

    This filter applies operators in each direction of the FFT space, enhancing edge and line structures.
    It uses local phase information, and not absolute intensity, which helps detecting weak features.

    It is almost finished. I have to do the eigenanalysis to the filter and contribute (again) to this open source library.

    ITK is the main library for image analysis in C++, specially useful in segmentation of medical images and 3D stacks.
\end{frame}

\begin{frame}[t]
  \frametitle{Segmentation after the enhancement of the monogenic filter}
     After enhancement with the monogenic filter two options:
     \begin{itemize}
       \item The monogenic signal phase can be used to detect lines and edges. Use this measure to detect edges (more reliable than gradient or Sobel). Then use a level-set method, where the edge-measure is used to set the external vector field constraining the evolution \citep{rajpoot_local-phase_2009}.
       \item Exploit monogenic filter steering (rotation) properties. This allows to set an Eigensystem to find at each pixel the direction that maximizes the signal. Then threshold the enhanced image to get the segmentation \citep{chenouard_3d_2011}.
     \end{itemize}
\end{frame}

\begin{frame}[t]
  \frametitle{Skeletonization of segmented image}
     After segmentation, we have to thin the resulting binary object to a one-pixel wide representation that conserves topology.
     
     Skeletonization is really sensitive to noise in the borders, so a pruning is always required.
     \begin{columns}[T,onlytextwidth]
       \column{0.5\textwidth}
       \begin{exampleblock}{Global pruning:}
         \begin{itemize}
           \item It applies after the skeleton has been extracted.
           \item Only takes into account length of branches removing short but thin branches that might be significant for the architecture.
         \end{itemize}
       \end{exampleblock}
       \begin{figure}[h]
       \includegraphics[height=2cm]{./Figures/skeleton/rectangle3D.jpg}
       \end{figure}
       \column{0.5\textwidth}
       \begin{exampleblock}{Local pruning:}
         \begin{itemize}
           \item Novel method to pruning at the same time than skeleton is generated.
           \item The pruning is made locally, thanks to a measure of \textbf{length of branch versus thickness.}
           \item It conserves short branches that are locally thin, but deletes short branches that are thick (see figure)
         \end{itemize}
       \end{exampleblock}
     \end{columns}
\end{frame}

\begin{frame}[t]
  \frametitle{Skeletonization: Cubical Complexes}
    \begin{figure}[h]
      \centering
      \includegraphics[height=1cm]{./Figures/software_logos/dgtalLogo.png}
      \caption*{DGTal:: Digital topology library}
    \end{figure}
    I have already contributed to DGtal the implementation of this local prune algorithm based on \citep{couprie_3d_2015}.

    And I am in the developing team!
    \begin{figure}[h]
      \includegraphics[width=0.6\textwidth]{./Figures/skeleton/dgtalGithub.png}
    \end{figure}
\end{frame}

\begin{frame}[t]
  \frametitle{Skeletonization: Cubical Complexes}
  \begin{columns}[T,onlytextwidth]
    \column{0.5\textwidth}
    \begin{exampleblock}{Global pruning:}
      \begin{itemize}
        \item Uses a framework called CubicalComplexes or VoxelComplexes, where a spel (voxel in 3D, pixel in 2D) is composed by points, lines and faces.
        \item This allow to define a topology property called Isthmus in all those components, that preserves topology in the skeletonization process.
        % \item The local prunning is made setting a life-time parameter, measuring the iterations between a pixel has been marked for removal (because its deletion doesn't change topology) and the iteration where it was definitely removed.
      \end{itemize}
    \end{exampleblock}
    \column{0.5\textwidth}
    \begin{figure}[h]
      \includegraphics[width=0.9\textwidth]{./Figures/skeleton/thinningCouprie.png}
    \end{figure}
  \end{columns}
\end{frame}

\begin{frame}[t]
  \frametitle{Summary of image analysis}
  \begin{exampleblock}{Pipeline:}
    \begin{description}[<+-| alert@+>]
      \item<1->[Multiscale analysis and denoise:] Detect vessels of different radius.
      \item<2->[Segmentation:] Get a binary object that conserves features. 
      \item<3->[Skeletonization:] One-pixel wide (thin) representation of binary object, conserving topology.
      \item<4->[Graph adaptor:] Transform skeleton to a graph, to connect with ComplexSystems tools for characterization.
    \end{description}
  \end{exampleblock}
  \begin{textblock*}{2cm}(1.5cm,6.9cm) % {block width} (coords)
    \tikzmark{n1}
    \uncover<2->{\includegraphics[width=2cm]{./Figures/fire/nucleation1.png}}
  \end{textblock*}
  \begin{textblock*}{1.5cm}(7.5cm,6.5cm)
    \tikzmark{t1}
    \uncover<3->{\includegraphics[width=1.5cm]{./Figures/skeleton/lind.png}}
  \end{textblock*}
  \begin{textblock*}{2cm}(9.1cm,7cm)
    \uncover<4->{\includegraphics[width=2cm]{./Figures/skeleton/aviz.png}}
  \end{textblock*}
  \begin{tikzpicture}[remember picture,overlay]
    %\path[draw=magenta,thick,->]<3-> ([yshift=3mm]n1) to ++(0,3mm) to [out=0, in=0,distance=2.5in] (t1);
    \path[draw=green,thick,->]<3-> ([xshift=2.2cm, yshift=-1cm]n1) -- ([xshift=-0.2cm,yshift=-1.4cm]t1);
  \end{tikzpicture}
\end{frame}

\begin{frame}
  \frametitle{Generating Software Tools II}
  \begin{columns}[onlytextwidth]
      \column{0.49\textwidth}
      \includegraphics[width=0.9\textwidth]{./Figures/chapter-reconstruct/networkN10000.png}
      \vspace{1cm}
      Reconstruct in-silico network from graph statistical distributions
      %\citep{lindstrom_biopolymer_2010}:
      \column{0.49\textwidth}
      \includegraphics[width=0.9\textwidth]{./Figures/software_screenshots/sgext.png}
      \vspace{1cm}
      Not revelead yet (sorry!): Merge of software when segmentation is ready.

      SGEXT: Spatial Graph Extractor. Name suggestions welcome! (MAIDEN?)
      %\citep{lindstrom_biopolymer_2010}:
  \end{columns}
\end{frame}


\end{document}

